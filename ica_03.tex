\documentclass[10pt]{article}
 \usepackage[margin=1in]{geometry} 
\usepackage{amsmath,amsthm,amssymb,amsfonts,color, titling}
 \usepackage{listings}

\setlength{\droptitle}{-20mm} 

\usepackage[colorlinks=true,linkcolor=red,urlcolor=blue]{hyperref}

\title{In-class assignment \# 3}
\author{Brian O'Shea, \\PHY-905-003, Computational Astrophysics and
  Astrostatistics\\Spring 2017}
 \date{} % leave blank to have no date

\begin{document}
 
\maketitle

\vspace{-5mm}

\noindent \textbf{Instructions:} We're going to use the functions you
wrote for the Bisection and Newton's method, as well as some of the
additional root finding methods in the
\href{https://docs.scipy.org}{SciPy}
\href{https://docs.scipy.org/doc/scipy-0.18.1/reference/optimize.html#root-finding}{\texttt{optimize}}
package (specifically, the methods for scalar functions).

Your goal is straightforward: try to come up with multiple functions
f(x) that have one or more roots in an interval $[a,b]$ and which
break one or more of the following methods: the Bisection Method,
Newton's Method, the Secant Method (which is Newton's method with a
numerically-computed derivative), and
\href{http://mathworld.wolfram.com/BrentsMethod.html}{Brent's Method}.
In this case, ``break'' simply means to force the method to not find
an existing root in that interval.

Experiment with several functions and all of the methods described
above.  What are the characteristics of functions or chosen intervals
$[a,b]$ that tend to break root-finding algorithms?  Are some
algorithms more robust than others?  Are any of these algorithms
effectively unbreakable?  Make some notes in \texttt{ANSWERS.md}, and
we'll also discuss it in class.

\vspace{5mm}

\noindent 
\textbf{What to turn in:} Turn in \texttt{ANSWERS.md}, any
source code you wrote, any plots you created (and the scripts you used
to create them).  \textbf{Do not} turn in object files or
executables!

\end{document}